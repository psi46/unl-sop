\documentclass[12pt]{unlsilabsop}
\title{Storage of raw, intermediate and finished parts}
\date{September 14, 2014}
\author{Frank Meier Aeschbacher}
\approved{Frank Meier Aeschbacher}
\sopid{004}
\sopversion{v0}
\sopabstract{Describes how parts used and produced during manufacturing are stored.}
\begin{document}

\maketitle

%------------------------------------------------------------------
\section{Scope}
This covers access to the facilities used at UNL to manufacture modules.

%------------------------------------------------------------------
\section{Purpose}
Raw parts and finished modules need special protection, including storage. This document describes how parts are stored safely.

%------------------------------------------------------------------
%>\section{Definitions}

%------------------------------------------------------------------
\section{Responsibilities}
All SiLab team members are required to obey these rules.

%------------------------------------------------------------------
\section{Principles}

\begin{tabular}{p{2.5cm}p{4.5cm}p{8cm}}
    \toprule
Objects & Production stage & Storage rules \\
    \midrule
BBM & Delivery & Containers (GelPaks) remain closed outside the cleanroom. Transfer to cleanroom as soon as feasible. \\
    & Until manufacturing & Store in cabinet inside cleanroom in labelled area. Keep containers closed until used. \\
    \midrule
HDI & Delivery & Containers (ESD boxes) remain closed outside the cleanroom. Transfer to cleanroom as soon as feasible. \\
    & Until manufacturing & Store in cabinet inside cleanroom in labelled area. Keep containers closed until used. \\
    \midrule
Module & Glued, wirebonded but not yet encapsulated & Store on chucks or mounted on module carriers (depending on following steps) in cabinet inside cleanroom in labelled area. \\
       & Encapsulated, mounted on module carrier & Store in cabinet inside cleanroom in labelled area. Module may leave cleanroom for testing and shipping. \\
    \bottomrule
\end{tabular}

\end{document}


\documentclass[12pt]{unlsilabsop}
\title{Module assembly: Deliveries of HDI}
\date{September 14, 2014}
\author{Frank Meier Aeschbacher}
\approved{Frank Meier Aeschbacher}
\sopid{101}
\sopversion{v0}
\sopabstract{Describes the procedures upon receiving if HDI. This includes tests performed, documentation and storage.}
\begin{document}

\maketitle

%------------------------------------------------------------------
\section{Scope}
This is a regular part in the manufacturing process of pixel modules at UNL. This process gets triggered upon delivery of a shipment with HDI.

%------------------------------------------------------------------
\section{Purpose}
HDI are a critical part for manufacturing modules. This procedures ensures the bookkeping of new deliveries and triggers required testing procedures. In case of issues this procedure triggers communication with suppliers.

%------------------------------------------------------------------
%>\section{Definitions}

%------------------------------------------------------------------
\section{Responsibilities}

\begin{itemize}
    \item Shipper: Informs us of any shipment underway (usually Fermilab is the shipper).
    \item Business office: Informs us of received shipments
    \item SiLab Team member: Picks up shipment and follows instructions. Keeps new deliveries separate from already handled deliveries until this procedure has been finished.
\end{itemize}

%------------------------------------------------------------------
\section{Equipment}

\begin{itemize}
    \item Probe station
    \item Microscope with USB camera attached and illumination
    \item Tweezers for HDI handling
\end{itemize}

%------------------------------------------------------------------
\section{Procedure}

The procedure below describes the steps for bonding one module. The chuck provides space for up to 4 modules, which may be processed in one batch at the discretion of the operator.
\begin{enumerate}
    \item Unpack delivery in receiving room but keep ESD boxes closed until transferred into cleanroom. Do not bring cardboard boxes into cleanroom.
    \item Handle HDI only with proper protection: ESD wristband, gloves, face mask.
    \item Open ESD boxes and get a first impression: Shipment should be in order and no immediate damage is visible.
    \item Take note of the serial numbers of HDI included in the delivery.
    \item Compare list of serial numbers with delivery statement. Report any inconsistency in documentation and inform shipper.
    \item Perform the visual inspection of HDI according to SOP~201.
    \item Perform the electrical acceptance test of HDI according to SOP~204.
    \item Failed HDI should be clearly marked as such and kept separate from good ones. Shipper needs to be informed and action taken.
    \item Document any findings.
\end{enumerate}

%------------------------------------------------------------------
\section{Documentation}
The following information needs to be recorded in the report for the UNL logbook:
\begin{itemize}
    \item Date, time (start--end) and operator name
    \item Id of delivery: date received, shipment/tracking number
    \item List of parts received
    \item Any special observations, e.g.~damage to parts, deviations from normal procedures
\end{itemize}
A template document to fill in is provided on the TWiki page and may be used as guidance.

Purdue database: Status of HDI needs to be updated.

%------------------------------------------------------------------
\section{References}

The manual to the wirebonder is available in the cleanroom as a printed copy in a folder and electronically as PDF on the cleanroom computer.

\end{document}


\documentclass[12pt]{unlsilabsop}
\title{Electrical acceptance test of assembled modules}
\date{February 5, 2016}
\author{Joaquin Siado}
\approved{Rachel Bartek}
\sopid{206}
\sopversion{v1}
\sopabstract{Describes the procedures for the electrical test of final modules. This consists of taking the IV curve and running a so-called \emph{full test} with the standardized test software. Any deviations from normal appearance will be documented.}
\newcommand{\at}{\makeatletter @\makeatother}
\begin{document}

\maketitle

%------------------------------------------------------------------
\section{Scope}
This is a regular test in the manufacturing process of pixel modules at UNL.

%------------------------------------------------------------------
\section{Purpose}
The electrical acceptance test of final modules is a critical step in manufacturing modules. This test is a critical control point to make sure the material we send out meets expectations and to ensure the quality.

%------------------------------------------------------------------
%>\section{Definitions}

%------------------------------------------------------------------
%\section{Responsibilities}

%------------------------------------------------------------------
\section{Equipment}

\begin{itemize}
\item \textbf{Cold box}  With a program (Fulltest) to keep the temperature at 17$^\circ$C for at least 4 hours.
\item \textbf{DTB with adapter cards} 4 each, adapter cards connected via SCSI ribbon cable. Power brick connected. All 4 connected to the computer via separate USB cables. Do not use a USB hub unless you are sure it works. 4 HV cables connected to the Keithly power supply. Power up a DTB only if it will be used.
\item \textbf{Computer} A Linux or Mac~OS~X based computer with pXar, ElCommandante and MoReWeb installed.
\item \textbf{Chiller} Model RC045, Serie J03
\item \textbf{Dry air supply} Set to 60 PSI
\item \textbf{HV power supply} Keithly model 2410, connected to the same computer through a USB to RS232 adapter
\end{itemize}

%------------------------------------------------------------------
\section{Procedure}

\begin{enumerate}
    \item Handle modules only with proper protection: ESD wristband.
    \item Check encapsulated drawer in the dry air cabinet for any module that need to be tested.
    \item Take up to four and place them in the cold box. For this, do:
    \begin{itemize}
        \item Open the lid. The sample area needs to be clean, no dust or dirt visible on cold plate.
        \item Place the modules using the positioning pins.
        \item Connect the cable to the adapter cards on the cold box.
        \item Close the lid. Make sure the cables lie flat and the lid makes a good seal.
    \end{itemize}    
    \item Turn on 	the cold box. %and run the Fulltest program. The temperature need to be stable, within .3$^\circ$C , before running module tests.
    \item Power on the needed DTBs.
    \item Go to FPIXUtils directory and do.
    \begin{itemize}
    		\item {$\$ $\it{python startShift.py}}
    		\item Enter your information and modules information.
    		\item {$\$ $\it{python prepareTest.py Pretest False}}
    		\item When the prestest is finished do: {$\$ $\it{python checkTest.py Pretest}}
    		\item Check all graphs. If modules need to be removed do so and run {$\$ $\it{startShift.py}} again with the new module info
    		\item If no modules were removed do: {$\$ $\it{python prepareTest.py FPIXTest False}}
    		\item When flushing part is over, start the chiller and set it to 20$^\circ$C. Water flow set to 3\,gal/min.
    		\item Go to elcomandante subfolder and run {$\$ $\it{python el\_comandante.py}}
    		\item When test is finished turn off the chiller and run {$\$ $\it{python checkTest.py FPIXTest}}
    \end{itemize}
    \item Removing the modules
   \begin{itemize}
   		\item Open the lid and remove the modules.
   		\item Prepare them for the next step, shipping them to KU if they are in good condition, shipping them to FermiLab if further testing (debugging) is needed.
	\end{itemize}
\end{enumerate}

%------------------------------------------------------------------
\section{Documentation}
\begin{enumerate}
	\item Purdue database
		\begin{itemize}
			\item Go to Purdue database, assembly status and click on encapsulated or wirebonded
			\item Click on the module name and read the sensor attached to it
			\item Then, go to main menu. Measurement submit, select the wafer, the last three digit in the sensor number
			\item Fill out the information (choose the log file containing the IV results) and submit
			\item Go to main menu, Moreweb data submit, and select the .tar file corresponding to the module
		\end{itemize}
	\item Github
		\begin{itemize}
			\item In a terminal window do: git clone https://github.com/jking79/UNL-KU-FPIX-Module-Testing
			\item Copy the entire folder of the modules to this folder, then do:
			\item git add .
			\item git commit -m "comments"
			\item git push origin master				 
		\end{itemize}
	\item If a module is encapsulated promote it to tested in purdue database.
	\item Go to Full test summary, then manual update, check full test at 17$^\circ$C, and set next testing step to x-ray.
\end{enumerate}
\end{document}


\documentclass[12pt]{unlsilabsop}
\title{Vacuum supply}
\date{August 4, 2015}
\author{Frank Meier Aeschbacher}
\approved{Frank Meier Aeschbacher}
\sopid{305}
\sopversion{v1}
\sopabstract{This document describes the regular maintenance for the vacuum supply.}
\begin{document}

\maketitle

%------------------------------------------------------------------
\section{Scope}
This document describes the maintenance of the vacuum supply.

%------------------------------------------------------------------
\section{Purpose}
The vaccum supply is crucial for running the gantry, the wirebonder and the probe station modules. This SOP describes the maintenance steps for keeping it in working condition.

%------------------------------------------------------------------
%\section{Definitions}

%------------------------------------------------------------------
\section{Responsibilities}
Every person using vacuum is responsible to maintain its supply. While a minimum schedule for maintenance activities is outlined in this document, any person is allowed to trigger out-of-cycle maintenance if needed to maintain working conditions.

%------------------------------------------------------------------
\section{Equipment}

\emph{No special equipment needed}
%\begin{itemize}
%    \item
%\end{itemize}

% Consider adding images of key equipment if it helps

%------------------------------------------------------------------
\section{Procedure}

% Mention to record certain observations if this is needed
\subsection{Weekly maintenance}
\begin{enumerate}
    \item Check the pumps for any accumulated dust. For this, turn the pumps off and remove the cover. Inspect the pumps and remove any dust or dirt. Reattach the cover and turn the pumps back on.
\end{enumerate}

%------------------------------------------------------------------
\section{Documentation}
The following information needs to be recorded in the report for the UNL logbook:
\begin{itemize}
    \item Date, time (start--end) and operator name.
    \item Any special observations, e.g.~any deviation from expectation.
    \item Any action taken.
\end{itemize}

\end{document}


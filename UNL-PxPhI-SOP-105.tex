\documentclass[12pt]{unlsilabsop}
\title{Module assembly: Encapsulation of wirebonds}
\date{September 22, 2014}
\author{Frank Meier Aeschbacher}
\approved{Frank Meier Aeschbacher}
\sopid{105}
\sopversion{v0}
\sopabstract{Describes the procedures to encapsulate wirebonds of modules using the robotic gantry. The procedure takes place in the clean room.}
\begin{document}

\maketitle

%------------------------------------------------------------------
\section{Scope}
This is a regular part in the manufacturing process of pixel modules at UNL.

%------------------------------------------------------------------
\section{Purpose}
Wirebonds should receive protection by applying an encapsulant. This is the final manufaturing step.

%------------------------------------------------------------------
%>\section{Definitions}

%------------------------------------------------------------------
\section{Responsibilities}

%------------------------------------------------------------------
\section{Equipment}

\begin{itemize}
    \item Gantry
    \item Nordson ESD dispenser (TODO: model no.)
    \item Chucks equipped with carrier chucks for modules
    \item Sylguard 186 two-component encapsulant. Consists of erastomer base and elastomer curing agent.
    \item Test tube to mix encapsulant
    \item Disposable syringes for measuring volumes: 1\,cc size for curing agent and 10\,cc size for base
    \item Disposable syringe for dispenser, 5\,cc size, Nordson ESD, part no.
    \item Dispenser needle, color code: lavender, Nordson ESD part no.
    \item Centrifuge
    \item Polyimide sheets, to cover chuck positions if not running a full batch of 4 modules
    \item Felt tip marker, to write batch number on chuck
    \item Curing oven
\end{itemize}

TODO: add image of setup. Make note that arrangement may vary as long as it matches the setup in the software

%------------------------------------------------------------------
\section{Procedure}

The procedure below is good for encapsulation of a larger number of modules than chust the 4 of one chuck.
\begin{enumerate}
    \item Handle bare modules only with proper protection: ESD wristband, gloves, face mask.
    \item Perform a readiness check of the gantry:
    \begin{enumerate}
	\item Test presence of vaccum on the gauge, record the pressure.
	\item Test presence of compressed air by engaging the dispenser
	\item Start LabView control software. Record version. TODO: note further checks
    \end{enumerate}
    \item Prepare encapsulant in lab outside the cleanroom:\label{enum:prepencapsulant}
    \begin{enumerate}
	\item Check expiration date of both ingredients. Replace if expired.
	\item Record batch number and expiration date of encapsulant.
	\item Take 0.1\,cc of curing agent using the 1\,cc syringe. Transfer to test tube
	\item Take 1.0\,cc of base elastomer using the 10\,cc sysringe. Transfer to the same test tube.
	\item Gently shake test tube (TODO: Mention exact procedure) and transfer liquid to dispenser syringe.
	\item Fill another test tube with water
	\item Place both tubes in centrifuge at opposite location to balance rotor.
	\item Centrifuge for 2\,min. Repeat until encapsulant shows no bubbles anymore.
	\item Transfer syringe into cleanroom
    \end{enumerate}
    \item Attach tube to syringe and insert syringe into holder at gantry head.
    \item Run the calibration procedure to determine the position of the needle tip.
    \item Make a manual test deposit on a disposal surface. Make sure the lines look nice and show no interruptions from bubbles. If not, go back to step \ref{enum:prepencapsulant} and use a new syringe.
    \item Pick modules from storage and identify them. Check that the parts satisfy the quality criteria by retrieving their information in the database. \label{enum:startenacpulation}
    \item Record the UNL id (\texttt{Nxxxyy}) for each module. Write down the number on the edge of the chuck for further identification.
    \item Adjust configuration in software to reflect positions in use.
    \item Run pattern recognition step. Check if locations found are sound.
    \item Run program. Supervise progress and stop in need, especially when modules are at risk.
    \item At the end of the full cycle, check visually if all wirebond foots are covered as needed (TODO: add a sketch for comparison). Use manual mode to correct for any missing encapsulant and record such special action.
    \item Transfer chuck to curing oven. Run the program until finished.
    \item Finished modules shall not be handled for at least 2~hours.
    \item Document all actions even if the curing time may not be over.
    \item If more than one chuck full of modules needs to be encapsulated, repeat from step \ref{enum:startenacpulation} as often as needed.
    \item Remove the syringe and dispose off.
\end{enumerate}

%------------------------------------------------------------------
\section{Documentation}
The following information needs to be recorded in the report for the UNL logbook:
\begin{itemize}
    \item Date, time (start--end) and operator name
    \item LabView software: version
    \item Id of parts used: UNL batch identification
    \item Batch number and expiration date of Sylguard components
    \item Any special observations, e.g.~damage to parts not already recorded during visual inspection, deviations from normal procedures
\end{itemize}
A template document to fill in is provided on the TWiki page and may be used as guidance.

Purdue database: Status of BBM and HDI need to be updated.

\end{document}

